\documentclass{article}
\usepackage[utf8]{inputenc}
\usepackage[english]{babel}
\usepackage{geometry}          
\geometry{a4paper, margin=1in} 
\usepackage{graphicx} 
\usepackage{titlesec}


\title{Summary report on Machine Learning}
\author{Rafał Górniak \& Piotr Hynasiński}
\date{\today}

\begin{document}

\maketitle

\vspace{1cm}

\section{Introduction}

This report provides an overview of machine learning, its methods, and applications. The peper is divided into six sections, first five
describe one by one each of the attitudes and functions taken into account. The last one presents results and conclusions as well as
our opinions concerning approprability.

\newpage
\section{K-Nearest neighbour}
In description of nearest neighbour approach the information about extracted (selected) features, about their normalization and about similarity measure should be provided.


\newpage
\section{Decision tree}
While presenting decision tree approach you should describe induction of single tree (e.g. what conditions were considered, how information gain was measured, when the induction process was stopped). This section should include an image showing how one induced tree (for one person) looks like.


\newpage
\section{Random forest}
While presenting random forest approach you should describe working principles of random forest ensemble (e.g. how the diversity of ensemble members was assured, what was the procedure of results aggregation). This section should include images showing how sample induced trees in the fores (for one person) look like.


\newpage
\section{Person similarity}
Describing person similarity approach report how people were compared and how the final movie evaluation was calculated.


\newpage
\section{Collaborative filtering}
Discussing collaborative filtering approach focus on the process of training and on methods ensuring the generalization abilities of the trained model (e.g. selection of feature space dimensionality).


\newpage
\section{Summary and results}
Presentation of results for every approach should be similar to make their comparison easier.


\end{document}